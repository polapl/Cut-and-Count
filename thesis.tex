\documentclass[12pt, oneside]{report}

\usepackage[left=2.5cm, top=2.5cm, bottom=2.5cm, right=2.5cm]{geometry}

\usepackage[utf8]{inputenc}
\usepackage[T1]{polski}
\usepackage[polish]{babel}
\usepackage{physics}
\usepackage{graphicx}

\begin{document}  
\thispagestyle{empty}
\begin{titlepage}
    \begin{center}

           \Large
    \textbf{Uniwersytet Jagielloński w Krakowie}\vspace{0.2cm}\\ Wydział Matematyki i Informatyki
               \vspace*{1cm}
               
         \vspace{3cm}
         \Large
          \textbf{Pola Kyzioł}\\\vspace{0.5cm}
         \normalsize Nr albumu: 1092406\\
             \vspace{2cm}
        \Huge
        \textbf{Tytuł pracy dyplomowej}
      
        \vspace{1.5cm}
        \normalsize
        Praca magisterska\\
        na kierunku Informatyka Analityczna\\ \vspace{0.15cm}
        
        \vfill
        \vspace{2cm}
       \begin{minipage}{1\textwidth}
\begin{flushright}
Praca wykonana pod kierunkiem\\
dr hab. Tomasz Krawczyk\\
Instytut Informatyki Analitycznej 
\end{flushright}
\end{minipage}
        
        \vspace{2cm}
        \begin{center}
      Kraków 2019
        \end{center}
    \end{center}
\end{titlepage}

\newpage 
 \thispagestyle{empty}
\vspace{2.5cm}
\begin{flushleft}
\large \textbf{Oświadczenie autora pracy}\vspace{0.6cm}\\
\end{flushleft}

\noindent Świadom odpowiedzialności prawnej oświadczam, że niniejsza praca dyplomowa została napisana przeze mnie samodzielnie i nie zawiera treści uzyskanych w sposób niezgodny z obowiązującymi przepisami.\\

\noindent Oświadczam również, że przedstawiona praca nie była wcześniej przedmiotem procedur związanych z uzyskaniem tytułu zawodowego w wyższej uczelni.
\vspace{2cm}
\begin{center}
\begin{tabular}{lr}
................................~~~~~~~~~~~~~~~~~~~~~~~~~~~~~~~~~~~~~~&
.......................................... \\
{~~~~Kraków, dnia} & {Podpis autora pracy~~~~}
\end{tabular}
\end{center}
\vspace{5cm}
\begin{flushleft}
\large \textbf{Oświadczenie kierującego pracą}
\end{flushleft}

\noindent Potwierdzam, że niniejsza praca została przygotowana pod moim kierunkiem i~kwalifikuje się do przedstawienia jej w postępowaniu o nadanie tytułu zawodowego.
\vspace{2cm}
\begin{center}
\begin{tabular}{lr}
................................~~~~~~~~~~~~~~~~~~~~~~~~~~~~~~~~~~~~~~&
............................................ \\
{~~~~Kraków, dnia} & {Podpis kierującego pracą~~}
\end{tabular}
\end{center}
\vfill

\newpage
\tableofcontents

\newpage
  	\chapter{Dekompozycja drzewowa}
  		\section{Definicja dekompozycji i szerokości drzewowej}

Dekompozycją drzewową grafu $G$ nazywamy parę $\mathcal{T} = (T, \{X_t : t \in V(T)\})$, gdzie $T$ jest drzewem, a $\{X_t : t \in V(T)\}$
zawiera zbiory wierzchołków grafu $G$ i spełnia następujące warunki:
\begin{itemize}
	\item{Dla każdej krawędzi $\{u, v\} \in E(G)$, istnieje węzeł $t \in V(T)$, taki że $u \in X_t$ i $v \in X_t$.}
	\item{Dla każdego wierzchołka $v \in V(G)$, zbiór $\{t \in V(T): v \in X_t \}$ jest poddrzewem drzewa $T$.}
\end{itemize}
Od tej pory wierzchołki grafu wyjściowego $G$ będą nazywane po prostu \emph{wierzchołkami}, natomiast węzły drzewa $T$ będą nazywane \emph{kubełkami}.
\newline\newline
Szerokość drzewowa dekompozycji drzewowej $\mathcal{T}$ jest zdefiniowana następująco: \newline $sd_\mathcal{T} = max_{t \in V(T)} \abs{X_t - 1}$. Natomiast szerokość drzewowa grafu $G$ jest minimalną szerokością drzewową wziętą po wszystkich możliwych dekompozycjach drzewowych $G$: \newline $sd_G = min \{sd_\mathcal{T}: \mathcal{T} \text{ jest dekompozycją drzewową }G\}$.
  		
  		\section{Ładna dekompozycja drzewowa}
Dla uproszczenia posługiwania się dekompozycją drzewową przy definiowaniu algorytmów dynamicznych, będziemy używać tzw. \emph{ładnej dekompozycji drzewowej}, która została po raz pierwszy wprowadzona przez Kloks \cite{kloks}.
\newline\newline
\emph{Ładna dekompozycja drzewowa} $\mathcal{T} = (T, \{X_t\}_{t \in V(T)})$ musi spełniać następujące warunki:
\begin{itemize}
	\item{$T$ jest ukorzenione.}
	\item{Każdy \emph{kubełek} $T$ ma co najwyżej dwoje dzieci.}
	\item{Jeśli \emph{kubełek} $t$ ma dwoje dzieci $p$ i $q$, wtedy $X_t = X_p = X_q$.}
	\item{Jeśli \emph{kubełek} $t$ ma jedno dziecko $p$, to $\abs{X_t} = \abs{X_p} + 1$ \emph{oraz} $X_p \subset X_t$ albo $\abs{X_t} = \abs{X_p} - 1$ \emph{oraz} $X_t \subset X_p$.}
\end{itemize}
Ponieważ w ładnej dekompozycji drzewowej, \emph{kubełki} różnią się od siebie o co najwyżej jeden \emph{wierzchołek}, każde przejście między jednym a drugim \emph{kubełkiem} odpowiada dokładnie jednej operacji na grafie wyjściowym $G$. Każdy \emph{kubełek} ma jeden z następujących pięciu typów:
\begin{itemize}
	\item{\texttt{WPROWADZAJĄCY v} - \emph{kubełek} ten ma o jeden \emph{wierzchołek} więcej niż jego jedyne dziecko: $X_p \cup \{v\} = X_t$. Każdy wierzchołek $v \in V(G)$, ma co najmniej jeden kubełek wprowadzający.}
	\item{\texttt{ZAPOMINAJĄCY v} - \emph{kubełek} o jednym wierzchołku mniej niż jedgo jedyne dziecko: $X_t \cup \{v\} = X_p$. Jego specjalnym reprezentantem jest korzeń. Dla każdego wierzchołka $v \in V(G)$, istnieje dokładnie jeden kubełek zapominający.}
	\item{\texttt{SCALAJĄCY} - jedyny \emph{kubełek} posiadający dwoje dzieci: $X_t = X_p = X_q$, scala dwa podgrafy o przecięciu $X_t$.}
	\item{\texttt{LIŚĆ} - dla $t$ będącego liściem: $X_t = \emptyset$.}
	\item{\texttt{UZUPEŁNIAJĄCY uv} - \emph{kubełek}, który nie pojawił się w pierwotnej definicji ładnej dekompozycji drzewowej, ale ułatwia definiowanie algorytmów operujących na dekompozycjach drzewowych. \emph{Kubełek} uzupełniający wprowadza krawędź $uv \in E(G)$ (uzupełnia krawędziami reprezentację grafu $G$ w drzewie $T$). \emph{Kubełek} $t$ \texttt{UZUPEŁNIAJĄCY} $uv$ zawiera oba \emph{wierzchołki} krawędzi: $u \in X_t$ i $v \in X_t$. Dla każdego $uv$ istnieje dokładnie jeden kubełek uzupełniający i - przyjmując bez straty ogólności $t(u)$ jest przodkiem $t(v)$ (gdzie $t(v)$ to najwyższy \emph{kubełek}, taki że $v \in X_{t(v)}$) - znajduje się on pomiędzy $t(v)$ a \texttt{ZAPOMINAJĄCY v}.}
\end{itemize}

daj tu przykład
		\section{Obliczanie dekompozycji drzewowej}


\newpage
  	\chapter{Klasyczne algorytmy dynamiczne}
    	\section{Drzewo Steinera}

Mamy dany nieskierowany graf $G$ oraz zbiór wierzchołków $K$ będący podzbiorem $V(G)$, $K \subset V(G)$. Wierzchołki te nazywane są terminalami.
Naszym zadaniem jest znalezienie dla grafu $G$ takiego jego spójnego podgrafu $H$, który zawiera wszystkie terminale i jego rozmiar jest minimalny.
Zakładamy, że mamy daną ładną dekompozycję drzewową grafu wyjściowego $G$ : $\mathcal{T} = (T, \{X_t\}_{t \in V(T)})$. Dodatkowo, dla uproszczenia samego algorytmu, przyjmujemy, że każdy \emph{kubełek} zawiera przynajmniej jeden terminal.

\centering
{
	\includegraphics[width=3cm]{images.jpg}
}

\newpage
	\begin{thebibliography}{9}
		\bibitem{kloks} 
			T. Kloks. 
			\textit{Treewidth. Computations and approximations}. 
			Lecture Notes in Computer Science, 842, 1994.
 
		\bibitem{einstein} 
			Albert Einstein. 
			\textit{Zur Elektrodynamik bewegter K{\"o}rper}. (German) 
			[\textit{On the electrodynamics of moving bodies}]. 
			Annalen der Physik, 322(10):891–921, 1905.
 
		\bibitem{knuthwebsite} 
			Knuth: Computers and Typesetting,
			\\\texttt{http://www-cs-faculty.stanford.edu/\~{}uno/abcde.html}
	\end{thebibliography} 



\end{document}
